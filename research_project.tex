\chapter{Project: A Theoretical Free Electron Laser (TeoFEL)}

\section{Pertinence et caractère stratégique du projet}

In the last years the European Community invested a large amount of money in the construction of the X-ray Free-Electron-Laser in Hamburg (XFEL). According to their view the XFEL  “will open up areas of research that were previously inaccessible. Using the X-ray flashes of the European XFEL, scientists will be able to map the atomic details of viruses, decipher the molecular composition of cells, take three-dimensional images of the nanoworld, film chemical reactions, and study processes such as those occurring deep inside planets.”  The construction of a such large facility has been motivated by the breathtaking advances in the field of ultrafast phenomena of the last decades thanks to the progresses in laser technology. In fact the laser duration has been constantly decreased, recently reaching the attosecond domain [Goulielmakis2008]. At the same time, the concentration of light in space and time can reach very high power-flux densities, where combined non-linear and non-adiabatic phenomena occur. 
The key aspect of such phenomena is the possibility of observing the real-time electronic and atomic dynamics following the excitation by means of a strong and ultra-fast laser pulse. In practice, this dynamics is connected to the relaxation of an highly excited state through several decay channels like, e.g., multi-electron-hole excitations, collective modes, phonon excitations, charge transfer, and even formation or breaking of chemical bonds and molecular rearrangements. These microscopic mechanisms, which are known to take place on extremely short timescales (from 10 to 1000 fs, see e.g.: [Cerullo2002, Kukura2005]), are the basis for numerous potential applications, from solar energy to nanotechnology, biochemistry and life science. 
Despite the great scientific interest and the relevant technological applications, we have still a poor understanding of several key aspects of the ultra-fast dynamics following the primary excitation pulse, the reason being the enormous delay in the theoretical and numerical development. Indeed, the dynamics of the initial ensemble of electron-hole pairs excited by the laser pulse represents a theoretical and numerical challenge, whose description in realistic materials is well beyond the capabilities of state-of-the-art first principles approaches, and up to now it has been mostly based on ad-hoc models and/or approximations. Finally, by reaching the attosecond timescale, experiments can approach a new temporal regime where the dressing-up of correlations is not instantaneous, thus making the gap with theory even more profound. 
From these premises, it is quite clear the urge for ground-breaking theoretical and numerical advancements, relying on the accuracy of ab-initio methods, to be coupled to cutting-edge experiments. The key goal of the present proposal is, indeed, a complete, coherent and accurate theoretical analysis of the different phenomena following the optical excitation of paradigmatic nanostructures, solids and molecular aggregates with ultra-fast and strong laser pulses. 
While we aim at building a unified numerical platform for the comprehension of ultrafast and non-linear phenomena, potentially applicable to a wide variety of materials, we here focus on few paradigmatic systems, which represent on their own hot topics of research. Ranging from weakly interacting molecules to bulk materials, these applications will also represent a unique possibility for the validation of new theoretical and numerical tools in very different regimes. 

\section{Objectifs scientifiques et technologiques}

State of the art
Simulation of the electron real-time dynamics is very challenge task. Many advances have been recently made in finite systems as molecules or cluster by means of Time-Dependent Density Functional Theory (TDDFT) [Calvayrac2000]. However for solids and  nanostructures the situation is more complicated.  At present there are very few groups in position to simulate the real-time electron dynamics  in extended systems. In particular two groups: the one of Yabana and Bertch developed an approach based on  TDDFT to simulate real-time response in solids coupled with a classical ions dynamics[Yabana2014], and our group where we coupled dynamics Berry phase with correlation cominig from Green's functions in such a way to simulate non-linear response of extended systems [Attaccalite2011].

\section{Theoretical advances}

The interpretation and, more importantly, the prediction of the results of the pump-probe  experiments represent, from the theoretical and numerical point of view, a formidable task. Here we will use Many-Body Perturbation Theory (MBPT) coupled with Dependent Functional Theory (DFT) to attack this problem.  For bulk materials MBPT has been shown to be one of the most efficient methods to include correlation effects in the study of electronic excitations.  Here we propose to use its extension to non equilibrium namely the Non-Equilibrium Green's Function (NEGF) theory, through the solution of the Kadanoff-Baym equations (KBE).  At present  KBE have been implemented only for elementary (or model) systems. The ambition of this proposal is to develop a computationally efficient code to solve the KBE in nanostructures, solids or organic molecules thus providing a general theoretical framework to support the experimental activity.
To this purpose, it is our intention to merge static Density-Functional-Theory (DFT) with NEGF to build a parameter free numerical tool which solves the KBE in a Kohn-Sham (KS) basis including the interaction with the atomic degrees of freedom. This new approach will be named the Ab-Initio NEGF theory (AiNEGF) and represents one ambitious key objective of the present proposal. The use of a DFT basis will make the AiNEGF approach universal, in contrast to all ad-hoc implementations of the KBE to model systems. This is what has de-facto prevented the theory to be predictive, thus restricting the simulations to a simple post validation of the experimental results. It is our ambition to remove this limit of the simulations and to create the basis for a really innovative theoretical and numerical scheme.
At present we have developed an approach to calculate electronic dynamics in presence of an external field, including static correlation effects that are at the origin of exciton formation. We plan to extend this formulation in different directions. We identify three important advances for the first two/three years and a long term part both described below.
In the first years we plain to extend our real-time dynamics in different directions (see Fig. 1):
For situation in which the nuclear motion remains in the harmonic regime, we will introduce a phononic Green's function and treat the electron-phonon interaction to lowest order in perturbation theory. This requires to generalize the KBE to include the phononic degrees of freedom and to solve the coupled KBE for the phonon and electron propagator.

We we will include relaxation effects deriving from the electron-electron dynamics. This effects are extremely important in the first femto-second of the excited state dynamics, because they are responsible of the fast electronic thermalization and plasmon dynamics. These scattering terms will be derived from NEGF and then approximated in such a way to have a numerical treatable problem.

At present we considered only spin-unpolarized systems. We would like to include spin-variable in our equation of motion in such a way to be able to threat systems in presence of spin-orbit coupling or spin-polarized materials. This will allow us to simulate a new class of materials containing heavier atoms and also defects. 


Beyond the advances proposed above we plain a long term part of the project including environment effects.
The idea is to take into account, via embedding techniques, the effect of leads,  nanojunctions and, more generally, of complex environmental conditions. In most works the contacts are modeled with finite size clusters and consequently the results are affected by spurious reflections already after few tens of femtoseconds. A reliable description of electronic relaxation in the contacting bulks requires an accurate description of the electronic coupling between the discrete molecular states and the dense manifold of highly de-localized electronic levels. This can efficiently be done using embedding (or down-folding techniques). One of the challenges  is to provide a full real-time characterization of ultra-fast processes taking place in open nanostructures under intense laser fields. This requires to build up a feasible computational framework which has the possibility to describe particle exchange and energy exchange with the contacts.

\section{Applications}

A first class of systems is given by low-dimensional semiconducting materials, such as: a) isolated and interacting carbon nanotubes (CNTs); b) single- and few-layer MoS2. Thanks to the graphene gold rush [Novoselov2011], both carbon-based and other semiconducting layered materials are receiving an increasing attention. In fact, the presence of a band gap can be exploited to complement graphene in applications that require thin transparent semiconductors, such as optoelectronics and energy harvesting. In addition, both systems exhibit strong electron-hole correlation, leading to the formation of tightly-bound excitons[Maultsch2005; Olsen2011]. This makes them ideal candidates to study transient correlation phenomena, such as quasi-particles and excitons dressing-up, by applying the methodological and computational tools developed within this project. 
A second  class of systems is given by organic or inorganic interfaces relevant for photovoltaic applications. We will mainly concentrate on: a) complex molecular heterojunctions, where fullerenes (C60), known for their acceptor character, are interleaved by polymers or other aromatic compounds (e.g. C60:P3HT and C60:HBC, see Figure 2), acting as electron donors; b) metal-organic molecules, characterized by one or more macrocyclic ligands, and by a central metal or group (e.g., Cu, Zn, Pb, Fe, Sn, TiO, Ru2, Si(OH)2, acting generally as an electron donor to the ligands), anchored to TiO2 or substrates. Our aim is to gain further insight into the factors determining the conversion efficiency, such as level alignment, electronic coupling and charge-transfer mechanism. In particular, we will devote particular attention to: (i) the effect of morphological modifications and/or of specific functionalization strategies, which can crucially alter electronic properties and level alignment; (ii) the role played by quantum coherence and vibrational coupling in determining the quantum yield of the charge transfer mechanism.

\section{Cohérence de la préproposition}
In conclusion, in this project we propose to study the electronic out-of-equilibrium ultra-fast dynamics in solid-state, nano-structured and organic materials by cutting-edge theoretical approaches and numerical tools. We aim to develop new theoretical and computational techniques that will go well beyond the state-of-the-art tools available at the moment to the scientific community. This will permit to fill the existing gap between theory and experiments by combining the different  expertise of the research units involved. We believe that the potential implications of the proposed experimental and theoretical approaches go well beyond the objectives of the present proposal. If successful, they will potentially constitute a revolutionary, ground-breaking approach to the problem of non-equilibrium excited state dynamics. 
The project will be carried out by the principal investigator Claudio Attaccalite in collaboration with the Valerio Olevano, Elena Cannuccia plus a PhD student. The principal investigator has a large experience in ab-initio calculation of excited states and in particular linear and non-linear response in solids states physics. Valerio Olevano is an expert of many-body perturbation theory(MBPT) and time-dependent density functional theory, (TDDFT) while Elena Cannuccia is an expert in electron-phonon coulping (EPC).   
The  Phd student will  work on the short terms part of the project in collaboration with the main coordinator and the other partners. While the long term part will be carried out the permanent persons.  Finally we would like to mention that  Claudio Attaccalite, Valerio Olevano and Elena Cannuccia belong to the European network ETSF (European Theoretical Spectroscopy Facility) and this project will clearly benefit from the collaboration with this network.
