\chapter{Conclusions} 
In this thesis we presented a new formalism to study non-linear response of solids and nanostructures. Our approach is based on a real-time solution of an effective Schr\"odinger equation. Our starting point is the parameter free Kohn-Sham Hamiltonian. Then we include correlation effects trought single-particle operators in the Hamiltonian. We consider three important effects that are known to affect  linear and non-linear response in solids: the GW correction to the Kohn-Sham band structure; the local field effects that are due to the density response in inhomogeneous materials; and the electron-hole interaction generated by the screened exchange. We found that inclusion of these three effects is crucial to reproduce and predict non-linear response. In particular in low dimensional materials the electron-hole interaction can double the SHG response. Finally we consider a more efficient approach derived from Density Functional Polarisation Theory, an extension of TD-DFT to periodic systems. We show that simple functionals that depend only from density and polarisation are able to catch large part of these correlation effects. Thanks to the strong efficiency of the TD-DFPT approach, it will be possible to extend this kind of simulations to a large spectra of materials and structures.\\
\section*{Future developments and perspectives} 
The coupling between Modern Theory of Polarisation and correlation effects described by means of NEGF or TD-DFPT was the successful outcome of this thesis. However the story does not end here. There are still some open questions that are waiting for an answer. In particular the inclusion of non empirical dephasing terms in the real-time dynamics, and the application of this approach to complex spectroscopic techniques. A correct dephasing can be obtained by means of non-equilibrium Green's function theory. However a key ingredient of our formalism is the phase difference generated by the external perturbation and this quantity cannot be easily obtained from Green's functions or other perturbtive approaches. This makes the coupling of these two worlds quite cumbersome. Some authors proposed rather complicated solutions to these problems, that didn't find a general application in the scientific community. Another possible route is to pass for the velocity gauge, but in this case  the non-locality of the many-body operators make the dynamics difficult to solve.
The second problem and/or prospective is the application of the present formalism to complex spectroscopic techniques as four-wave-mixing, Fourier-spectroscopy and two-dimensional spectroscopy. These techniques requires multiple laser sources and  a deeper analysis to extract data from real-time simulations. For these reasons we are currently working on new techniques as {\it wave-let analysis} and {\it compress sensing} to reduce simulation time and access to new phenomena in an efficient way.
Finally there all the numerical implementation and code parallelization that we did not discuss in the present manuscript. At present we created a very efficient code for the non-linear response and we hope it will be released under GPL licence at the beginning of the next year.
